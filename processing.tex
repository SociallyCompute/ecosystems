\chapter{Data Processing}
\chapterauthor{Georgios Gousios}

Software engineering is an exceedingly data abundant activity. Nearly all
artifacts of a software development project, both static, such as source code,
software repositories and issues and dynamic, such as run-time logs and user
activity streams, contain information valuable for understanding and optimising
both the software products and the processes that created them. Unfortunately,
modern organisations often do not utilise this
wealth of information as a feedback and decision support instrument. Rather,
teams adopt new software development practices and employ the latest and
greatest technology, without questioning the results of their decisions.
Consequently, this leads to decisions that are often unnecessarily suboptimal
or downright wrong~\cite{Strigini:1996:IntuitionBasedDecisions}.

\section{Streaming Analytics}

Despite the existence of tools and methods for extracting data from software
development processes, products and ecosystems, one key aspect is missing:
real-time operation. To compensate for this deficiency, we need to come
up with
tools and methods to collect, query, aggregate, and summarise ecosystem data
as streams, enabling software practitioners to use analytics as a
core feedback loop. In the context of software engineering, we expect that
streaming software analytics will enable software practitioners to move beyond
information toward actionable insight, hence allowing them for an increase of
software process and system technical quality. \vspace{0.5mm}

\section{A research agenda}

With the proposed research, we aim at utilising the wealth of information
produced by distinct software development artifacts in order to enable software
practitioners to use software analytics as a feedback and decision support
instrument. Realising streaming software analytics, therefore requires an
understanding of distinct stakeholder information needs, and decisions they
influence. But more importantly, we strive to understand how those needs map to
software analyses. We believe that community co-ordinated efforts can help
realizing the streaming software analytics vision in at least the following
ways:

\begin{description}

\item[Requirements for analytics] Researchers need to identify the
  stackeholders' information needs by means of qualitative research. Early works
  by Buse and Zimmerman~\cite{Buse:2012:INS:2337223.2337343} and Begel and
  Zimmerman~\cite{Begel.Zimmerman_AnalyzeThis:2013} are on this direction, but
  need to be revisited in the light of real-time analytics and instant feedback.

\item[Infrastructure work] Developing analytics pipelines is a complex and error
  prone task. It is however an area of intense competition (both academic and
  industrial) and there is ample room for improvement. Software engineering
  researchers should work together with researchers in other fields (e.g.
  databases, programming languages) to tailor existing analytics systems to
  software engineering requirements.

\item[Feedback-driven ecosystem evolution] Software engineering development
  methods (e.g. Scrum) are currently based on much folklore and little
  evidence. As researchers, we should aim to enable feedback loops within the
  software engineering practice. Developers should be able to easily execute
  experiments (e.g. A/B tests), collect and correlate the results of applying
  specific design and development decisions and their outcomes. This way, teams
  and organizations can organically evolve their tooling and optimize their
  processes based on data-driven decisions rather than black-box methodologies.

% Dominic: any other ideas?

\end{description}
